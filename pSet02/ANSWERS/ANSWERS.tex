\documentclass[12pt]{article}

\usepackage{breqn}
\usepackage[margin=1in]{geometry} 
\usepackage{amsmath,amsthm,amssymb,enumitem}
\usepackage[german,spanish,english]{babel}
\usepackage{tensor}
\usepackage{graphicx}
\usepackage{esint}
\usepackage[T1]{fontenc}
\usepackage{mathtools}
\usepackage{siunitx}
\newenvironment{ex}[2][Exercise]{\begin{trivlist}
\item[\hskip \labelsep {\bfseries #1}\hskip \labelsep {\bfseries #2.}]}{\end{trivlist}}

\newenvironment{sol}[1][Solution]{\begin{trivlist}
\item[\hskip \labelsep {\bfseries #1:}]}{\end{trivlist}}

\newcommand{\meq}{\overset{!}{=}}
\DeclarePairedDelimiter\bra{\langle}{\rvert}
\DeclarePairedDelimiter\ket{\lvert}{\rangle}
\DeclarePairedDelimiterX\brk[2]{\langle}{\rangle}{#1\,\delimsize\vert\,\mathopen{}#2}
\DeclareSIUnit\angstrom{\text {Å}}

%DECLARATION OF DELIMITERS%

\DeclarePairedDelimiter\vb{\lvert}{\rvert}
\DeclarePairedDelimiter\rb{(}{)}
\DeclarePairedDelimiter\sqrb{[}{]}
\DeclarePairedDelimiter\cb{\{}{\}}
\DeclarePairedDelimiter\ab{\langle}{\rangle}
\DeclarePairedDelimiter\db{\|}{\|}


\begin{document}
\noindent Richard Abele \hfill \today \\[30pt]
\centerline{ \Large{ \textbf{ Numerical Methods in Physics and Astrophyiscs }}}
\centerline{ \Large{ \textbf{ Problem Set 2 - Problem 4: Fractals Through Newton-Raphson }}}

To fulfill the necessary requirements for the assignment, the root finding program from the previous problem set is adapted to work with complex numbers and find the roots of a complex function. Two loops are then created to iterate what will be the pixels of a fractal image. 

An initial program called \texttt{calc_data.c} is then created to log the requested data in a \texttt{.csv} file. This data is in a nicely readable format, showing \(x_{0}, y_{0}, k\rb*{z}, f\rb*{z}\), and \(\log_{10}\rb*{n}_{\text{iterations} }\) as noted in the assignment. In order to create the final fractal images though, it is more practical to create a file listing only the necessary parameters and no additional strings, hence the creation of a slightly modified program called \texttt{fractal.c}.

Fractals were then plotted with gnuplot, using the \texttt{.gp} files. Images in the form of png follow the format \texttt{fractal_plot-<equation ID>_<image version>} where the equation IDs are used to group the equations. 



\end{document}

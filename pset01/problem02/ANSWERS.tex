\documentclass[12pt]{article}

\usepackage{breqn}
\usepackage[margin=1in]{geometry} 
\usepackage{amsmath,amsthm,amssymb,enumitem}
\usepackage[german,spanish,english]{babel}
\usepackage{tensor}
\usepackage{graphicx}
\usepackage{esint}
\usepackage[T1]{fontenc}
\usepackage{mathtools}
\usepackage{siunitx}
\newenvironment{ex}[2][Exercise]{\begin{trivlist}
\item[\hskip \labelsep {\bfseries #1}\hskip \labelsep {\bfseries #2.}]}{\end{trivlist}}

\newenvironment{sol}[1][Solution]{\begin{trivlist}
\item[\hskip \labelsep {\bfseries #1:}]}{\end{trivlist}}

\newcommand{\meq}{\overset{!}{=}}
\DeclarePairedDelimiter\bra{\langle}{\rvert}
\DeclarePairedDelimiter\ket{\lvert}{\rangle}
\DeclarePairedDelimiterX\brk[2]{\langle}{\rangle}{#1\,\delimsize\vert\,\mathopen{}#2}
\DeclareSIUnit\angstrom{\text {Å}}

%DECLARATION OF DELIMITERS%

\DeclarePairedDelimiter\vb{\lvert}{\rvert}
\DeclarePairedDelimiter\rb{(}{)}
\DeclarePairedDelimiter\sqrb{[}{]}
\DeclarePairedDelimiter\cb{\{}{\}}
\DeclarePairedDelimiter\ab{\langle}{\rangle}
\DeclarePairedDelimiter\db{\|}{\|}


\begin{document}
\noindent Richard Abele \hfill \today \\[30pt]
\centerline{ \Large{ \textbf{ Numerical Methods in Physics and Astrophysics }}} 
\centerline{ \Large{ \textbf{ Problem Set 1; Problem 2}}}

We are given a differential equation of the form
\begin{align*}
	y'' + \lambda^{2} y & =  0. &\\
\end{align*}

In order to find the Eigenvalues of the given equation, we will first solve it analytically as suggested. 

A general solution to this type of differential equation is give by 
\begin{align*}
	y\rb*{x} & =  A \cos \rb*{\lambda x} + B \sin \rb*{\lambda x} &\\  
\end{align*}
with \(A,B \in \mathbb{R}\) and determined by the boundary conditions. 

The given boundary conditions are 
\begin{enumerate}[label=(\alph*)]
	\item \(y\rb*{0} = 0\)
	\item and \(y\rb*{1} = y' \rb*{1}\).
\end{enumerate}

From boundary condition \(\rb*{a}\) it follows that:
\begin{align*}
	y \rb*{0} & =  A \cos \rb*{ 0 } + B \sin \rb*{ 0} = A  = 0&\\  
	\Rightarrow y\rb*{x} & =  B \sin \rb*{ \lambda x} &\\ 
	\Rightarrow y' \rb*{x} & =  B \lambda \cos \rb*{ \lambda x} &\\ 
\end{align*}
We can now apply the second boundary condition as follows:
\begin{align*}
	y \rb*{1} & =  B \sin\rb*{\lambda} &\\ 
	y' \rb*{1} & =  B \lambda \cos \rb*{ \lambda} &\\
	\Rightarrow B \sin \rb*{ \lambda} & =  B \lambda \cos \rb*{\lambda} &\\  
	\Rightarrow \sin \rb*{ \lambda} & =  \lambda \cos \rb*{ \lambda} &\\  
\end{align*}
This equation can then be rearranged and solved using the Newton Method from Problem 1 (see parent directory). 
\begin{align*}
	\lambda \cos \rb*{ \lambda} - \sin \rb*{ \lambda} & =  0 &\\  
\end{align*}

Looking at a plot of the function in mathematica confirms that it is an oscillating odd function with infinite solutions. 

The first non-zero solution yielded with seven iterations of the newton method is
\begin{align*}
	x_{n} & =  4.493409 &\\
\end{align*}



\end{document}

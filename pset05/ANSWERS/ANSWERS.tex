\documentclass[12pt]{article}

\usepackage{breqn}
\usepackage[margin=1in]{geometry} 
\usepackage{amsmath,amsthm,amssymb,enumitem}
\usepackage[german,spanish,english]{babel}
\usepackage{tensor}
\usepackage{graphicx}
\usepackage{esint}
\usepackage[T1]{fontenc}
\usepackage{mathtools}
\usepackage{siunitx}
\newenvironment{ex}[2][Exercise]{\begin{trivlist}
\item[\hskip \labelsep {\bfseries #1}\hskip \labelsep {\bfseries #2.}]}{\end{trivlist}}

\newenvironment{sol}[1][Solution]{\begin{trivlist}
\item[\hskip \labelsep {\bfseries #1:}]}{\end{trivlist}}

\newcommand{\meq}{\overset{!}{=}}
\DeclarePairedDelimiter\bra{\langle}{\rvert}
\DeclarePairedDelimiter\ket{\lvert}{\rangle}
\DeclarePairedDelimiterX\brk[2]{\langle}{\rangle}{#1\,\delimsize\vert\,\mathopen{}#2}
\DeclareSIUnit\angstrom{\text {Å}}

%DECLARATION OF DELIMITERS%

\DeclarePairedDelimiter\vb{\lvert}{\rvert}
\DeclarePairedDelimiter\rb{(}{)}
\DeclarePairedDelimiter\sqrb{[}{]}
\DeclarePairedDelimiter\cb{\{}{\}}
\DeclarePairedDelimiter\ab{\langle}{\rangle}
\DeclarePairedDelimiter\db{\|}{\|}


\begin{document}
\noindent Richard Abele \hfill \today \\[30pt]
\centerline{ \Large{ \textbf{ Numerical Methods - Problem Set 5 }}}

\begin{ex}
	1 Central Difference Relation
\end{ex}

For \texttt{problem 1} we are asked to prove the following central difference relation: 
\begin{align*}
	y \rb*{ x_{0}} '' & =  y_{0}'' = \frac{y_{1} - 2 y_{0} + y_{-1}}{h^{2}} - \frac{h^{2}}{12} y^{\rb*{4}} \rb*{\zeta} &\\
\end{align*}
and asked to calculate the explicit form of the error term \(\mathcal{O } \rb*{h^{2}}\).

\begin{sol}  \end{sol}


In the lecture notes we are given the hint that central difference relations can be calculated by the appropriate combination of Taylor expansions around a given point. 

Knowing this, we begin by Taylor expanding \(y\rb*{x}\) around the point \(x\), specifically at \(x_{1} = x_{0} + h\) and \(x_{-1} = x_{0} - h\), right and left of the desired point. Note that \(h = x_{1} - x_{0} = x_{0} - x_{-1}\).

\begin{align*}
	y \rb*{ x_{0} + h } & =  y \rb*{ x_{0}} + h y' \rb*{x_{0}} + \frac{h^{2}}{2!} y'' \rb*{x_{0}} + \frac{h^{3}}{3!} y^{\rb*{3}} \rb*{x_{0}} + \frac{h^{4}}{4!} y ^{\rb*{4}} \rb*{\zeta} &\\
	y \rb*{ x_{0} - h} & =  y \rb*{ x_{0}}
	- h y' \rb*{x_{0}}
	+ \frac{h^{2}}{2!} y'' \rb*{x_{0}}
	- \frac{h^{3}}{3!} y^{\rb*{3}} \rb*{x_{0}}
	+ \frac{h^{4}}{4!} y^{\rb*{4}} \rb*{\zeta} &\\
\end{align*}

Adding these two equations together to isolate \(y''\rb*{x}\) yields the following:
\begin{align*}
	y \rb*{x_{0} + h} + y \rb*{x_{0} - h} & = 
	2 y \rb*{x_{0}} + h^{2} y'' \rb*{x_{0}} 
	+ \frac{h^{4}}{12} \rb*{ y^{\rb*{4}} \rb*{\zeta}} &\\
\end{align*}

% Similarly, subtracting yields:
% \begin{align*}
% 	y \rb*{x_{0} + h} - y \rb*{ x_{0} - h} & =  
% 	2 h y' \rb*{x_{0}}
% 	+ \frac{2 h^{3}}{6} y^{\rb*{3}} \rb*{ x_{0}} &\\
% \end{align*}

Rearranging this expression results in:
\begin{align*}
	y''\rb*{x_{0}} & =  \frac{y \rb*{x_{0} + h} - 2 y \rb*{ x_{0}} + y \rb*{ x_{0} - h }}{h^{2}}
	- \frac{h^{2} }{12} y^{\rb*{4}} \rb*{\zeta} &\\
	\Rightarrow y'' \rb*{x_{0}} & = 
	\frac{y_{1} - 2 y_{0} + y_{-1}}{h^{2}} 
	- \frac{h^{2}}{12} y ^{\rb*{4}} \rb*{\zeta} &\\
	\Box
\end{align*}

\newpage

\begin{ex}
	2 Central Difference Relation
\end{ex}

For \texttt{problem 2} we are asked to prove the following central difference relation:

\begin{align*}
	y_{0}'' & =  \frac{
		-y_{2} + 16 y_{1} - 30 y_{0} + 16 y_{-1} - y_{-2}
	}{12 h^{2}}
	+ \mathcal{O} \rb*{h^{4}} &\\
\end{align*}

and asked to calculate the explicit form of the error term \(\mathcal{O } \rb*{h^{4}}\).

\begin{sol}  \end{sol}

We once again begin by Taylor expanding around \(x\), this time at the points up to \(2h \) to the right and left.

\begin{align*}
	y_{1} & =  y_{0} + h y_{0}' + \frac{h^{2}}{2} y_{0}''
	+ \frac{h^{3}}{6} y_{0}^{\rb*{3}}
	+ \frac{h^{4}}{24} y_{0}^{\rb*{4}} + \mathcal{O} \rb*{h^{5}} &\\
	y_{-1} & = y_{0} - h y_{0}' + \frac{h^{2}}{2} y_{0}''
	- \frac{h^{3}}{6}y_{0}^{\rb*{3}}
	+ \frac{h^{4}}{24} y_{0}^{\rb*{4}} + \mathcal{O} \rb*{h^{5}} &\\
	y_{2} & =  y_{0} + 2h y_{0}' + \frac{\rb*{2h}^{2}}{2} y_{0}''
	+ \frac{\rb*{2h}^{3}}{6} y_{0}^{\rb*{3}}
	+ \frac{\rb*{2h}^{4}}{24} y_{0}^{\rb*{4} } + \mathcal{O} \rb*{h^{5}} &\\
	y_{-2} & =  y_{0} - 2h y_{0}' + \frac{\rb*{2h}^{2}}{2} y_{0}''
	- \frac{\rb*{2h}^{3}}{6} y_{0}^{\rb*{3}}
	+ \frac{\rb*{2h}^{4}}{24} y_{0}^{\rb*{4} } + \mathcal{O} \rb*{h^{5}} &\\
\end{align*}

After we perform these expansions, we substitute \(y_{1},y_{-1},y_{2},\) and \(y_{-2}\) back into the numerator of the desired relation:

\begin{align*}
	\text{Numerator}  = & - y_{2} + 16 y_{1} - 30y_{0} + 16 y_{-1} - y_{-2} &\\
	= & - \rb*{
y_{0} + 2h y_{0}' + \frac{\rb*{2h}^{2}}{2} y_{0}''
	+ \frac{\rb*{2h}^{3}}{6} y_{0}^{\rb*{3}}
	+ \frac{\rb*{2h}^{4}}{24} y_{0}^{\rb*{4} } + \mathcal{O} \rb*{h^{5}} 
	} &\\
	& + 16 \rb*{
y_{0} + h y_{0}' + \frac{h^{2}}{2} y_{0}''
	+ \frac{h^{3}}{6} y_{0}^{\rb*{3}}
	+ \frac{h^{4}}{24} y_{0}^{\rb*{4}} + \mathcal{O} \rb*{h^{5}} 
	} &\\
	& - 30 \rb*{
		y_{0}
	} &\\
	& + 16 \rb*{
y_{0} - h y_{0}' + \frac{h^{2}}{2} y_{0}''
	- \frac{h^{3}}{6}y_{0}^{\rb*{3}}
	+ \frac{h^{4}}{24} y_{0}^{\rb*{4}} + \mathcal{O} \rb*{h^{5}} 
	} &\\
	& - \rb*{
y_{0} - 2h y_{0}' + \frac{\rb*{2h}^{2}}{2} y_{0}''
	- \frac{\rb*{2h}^{3}}{6} y_{0}^{\rb*{3}}
	+ \frac{\rb*{2h}^{4}}{24} y_{0}^{\rb*{4} } + \mathcal{O} \rb*{h^{5}} 
	} &\\
\end{align*}

As can trivially be seen, all terms of order \(y_{0}, y_{0}',\) and \(y_{0}^{\rb*{3}}\) cancel out, leaving
\begin{align*}
	\text{Numerator} & = y_{0}'' \rb*{-2 h^{2} + 8 h^{2} + 8 h^{2} - 2 h^{2}} &\\
& \text{       } + y_{0}^{\rb*{4} } \rb*{ \frac{-16 h^{4}}{24} + \frac{16 h^{4}}{24} + \frac{16 h^{4}}{24} + \frac{-16 h^{4}}{24}} + \mathcal{O} \rb*{h^{5}}&\\
	& = 12 h^{2} y_{0}'' + 0 y_{0}^{\rb*{4}} + \mathcal{O} \rb*{ h^{5}} &\\
	& = 12 h^{2} y_{0}'' + \mathcal{O} \rb*{h^{5}} &\\
\end{align*}

As we can see from this calculation, error terms of the order \(h^{4}\) cancel out, leaving us only with terms of the order \(h^{5}\) and higher.

Dividing the above calcualted numerator by the denominator \(12 h^{2}\) results in the desired equality. 


\newpage

\begin{ex}
	3 Numerical Differentiation of Arctan (to be addressed at a later time)
\end{ex}

We are given the function \(f\rb*{x}  = \tan ^{-1} \rb*{x} \) (assumed to be \( \arctan \rb*{x}  \)) and asked to calculate the derivative at \(x = 0\) using 

\begin{enumerate}[label=(\alph*)]
	\item two iterations of Richardson extrapolation and compare the results with those obtained from the central difference relations.
	\item splines and compare the results with those obtained from the central difference relations.
\end{enumerate}

\begin{sol}  \end{sol}

For a basis of comparison, we know that the analytical derivative of \(\arctan \rb*{x} \) is given by \(\frac{1}{1 + x^{2}}\), which will yield \(f'\rb*{0} = 1\).

\begin{enumerate}[label=(\alph*)]
	\item 
		We know that the first order central difference approximation is given by 
        \begin{align*}
            f'\rb*{x}  & \approx \frac{f \rb*{x + h} - f \rb*{x - h}}{2h} &\\
            \Rightarrow f' \rb*{0} & \approx \frac{f \rb*{h} - f \rb*{ -h}}{2h} &\\
            \Rightarrow f' \rb*{0} & \approx \frac{\arctan\rb*{h} - \arctan \rb*{-h}  }{2h} &\\
            \Rightarrow f' \rb*{0} & \approx \frac{\arctan \rb*{h} }{h} &\\
        \end{align*}
        From the lecture notes we know that the first order Richardson extrapolation has the form
        \begin{align*}
            y_{0}' & = \frac{1}{3} \sqrb*{ 4 F \rb*{h/2} - F \rb*{h}} &\\
        \end{align*}
        with \(F\rb*{h}\) be given by the first order central difference, 
        while the second order term has the form
        \begin{align*}
            y_{0}' & =  \frac{1}{15} \sqrb*{16 \tilde{F} \rb*{ h/2} - \tilde{F} \rb*{h}} &\\
        \end{align*}
        with \(\tilde{F} \rb*{h}\) corresponding to the first order Richardson approximation at \(h\).

        For explicit calculations, we now let \(h = 0.2\).

        \begin{itemize}
            \item True Value:
                \begin{align*}
                    y_{0}' & = 1 &\\
                \end{align*}
            \item Central Difference:
                \begin{align*}
                    y_{0}' & = \frac{\arctan \rb*{h} }{h} &\\
                    & = \frac{\arctan \rb*{0.2}  }{0.2} &\\
                    & = 0.986978 &\\
                \end{align*}
            \item First Order Richardson Extrapolation:
                \begin{align*}
                    y_{0}' & = \frac{1}{3} \sqrb*{ \frac{4 \arctan \rb*{h/2} }{h/2} - \frac{\arctan \rb*{h} }{h}} &\\ 
                    & = \frac{1}{3} \sqrb*{
                        \frac{4 \arctan\rb*{0.1} }{0.1} - \frac{\arctan0.2 }{0.2}
                    }  &\\
                    & = 0.999923  &\\
                \end{align*}
            \item Second Order Richardson Extrapolation:
                \begin{align*}
                    y_{0}' & = \frac{1}{15} \sqrb*{16 \tilde{F} \rb*{ h / 2} - \tilde{F} \rb*{h}} &\\
                    & = \frac{1}{15} \sqrb*{
                        \frac{16}{3} \sqrb*{\frac{4 \arctan \rb*{ h / 4} }{h /4} - \frac{\arctan \rb*{h/2} }{h/2}}
                        - \frac{1}{3} \sqrb*{ \frac{4 \arctan \rb*{h/2} }{h/2} - \frac{\arctan \rb*{h} }{h}}
                } &\\
                & = 0.999999862780083 &\\
                \end{align*}
        \end{itemize}
        As can be seen by the required increase in the number of significant digits, the second degree Richardson approximation is far more accurate than the simple central difference or even the first order Richardson approximation. 
    \item 
        To calculate the derivative of \(f \rb*{x}\) using a spline, we must first fit a spline of the following form to the equation:
        \begin{align*}
            S \rb*{x} & =  a x^{3} + b x^{2} + c x + d &\\
        \end{align*}
        We elected to use a cubic spline, meaning that four data points are required. 

        We chose the four values of \(x = {-h2, -h, h, 2h}\), yielding the following system of equations.
        \begin{align*}
            f \rb*{- 2h} & =  a \rb*{-2h}^{3} + b \rb*{-2h}^{2} + c \rb*{-2h} + d &\\
            f \rb*{- h} & =  a \rb*{-h}^{3} + b \rb*{-h}^{2} + c \rb*{-h} + d &\\
            f \rb*{h} & =  a \rb*{h}^{3} + b \rb*{h}^{2} + c \rb*{h} + d &\\
            f \rb*{ 2h} & =  a \rb*{2h}^{3} + b \rb*{2h}^{2} + c \rb*{2h} + d &\\
        \end{align*}
        We then procede by plugging in a value of \(h = 0.2\) and using Mathematica to solve the system of equations. 
        \begin{align*}
            a & =  -0.297599 &\\
            b & =  0 &\\
            c & =  0.998882 &\\
            d & =  0 &\\
        \end{align*}
        If we take the derivative of the spline
        \begin{align*}
            S' \rb*{x} & =  3 a x^{2} + 2 b x + c &\\
        \end{align*}
        and set \(x = 0\) to find the derivative at the origin
        \begin{align*}
            S' \rb*{0} & =  c, &\\
        \end{align*}
        we see that \(c\) must correspond to the derivative at \(x = 0\), thus the spline method yields \(f_{0}'  = 0.998882\), which is more accurate than the simple central difference method, but still less accurate than the Richardson extrapolation. 
\end{enumerate}


\end{document}

\documentclass[12pt]{article}

\usepackage{breqn}
\usepackage[margin=1in]{geometry} 
\usepackage{amsmath,amsthm,amssymb,enumitem}
\usepackage[german,spanish,english]{babel}
\usepackage{tensor}
\usepackage{graphicx}
\usepackage{esint}
\usepackage[T1]{fontenc}
\usepackage{mathtools}
\usepackage{siunitx}
\newenvironment{ex}[2][Exercise]{\begin{trivlist}
\item[\hskip \labelsep {\bfseries #1}\hskip \labelsep {\bfseries #2.}]}{\end{trivlist}}

\newenvironment{sol}[1][Solution]{\begin{trivlist}
\item[\hskip \labelsep {\bfseries #1:}]}{\end{trivlist}}

\newcommand{\meq}{\overset{!}{=}}
\DeclarePairedDelimiter\bra{\langle}{\rvert}
\DeclarePairedDelimiter\ket{\lvert}{\rangle}
\DeclarePairedDelimiterX\brk[2]{\langle}{\rangle}{#1\,\delimsize\vert\,\mathopen{}#2}
\DeclareSIUnit\angstrom{\text {Å}}

%DECLARATION OF DELIMITERS%

\DeclarePairedDelimiter\vb{\lvert}{\rvert}
\DeclarePairedDelimiter\rb{(}{)}
\DeclarePairedDelimiter\sqrb{[}{]}
\DeclarePairedDelimiter\cb{\{}{\}}
\DeclarePairedDelimiter\ab{\langle}{\rangle}
\DeclarePairedDelimiter\db{\|}{\|}


\begin{document}
\noindent Richard Abele \hfill \today \\[30pt]
\centerline{ \Large{ \textbf{ Numerical Methods - Problem Set 6 }}}

\begin{ex}
    1 
\end{ex}
In \texttt{problem 1} we are asked to determine which formula from \texttt{table 1} in the lecture notes on numerical integration will analytically be derived from combining the Simpson \(\frac{1}{3}\) integration method with the Romberg integration method. 

The five formulas in the lecture note table are as follows: 

\begin{align}
    \int_{x_{0}}^{x_{1}} {f \rb*{x}} \,d x & = \frac{h}{2} \rb*{f_{0} + f_{1}} - \frac{1}{12} h^{3} f^{\rb*{2}} \rb*{ \zeta_{1}} &\\
    \int_{x_{0}}^{x_{2}} {f \rb*{x}} \,d x & = 
    \frac{h}{3} \rb*{f_{0} + 4 f_{1} + f_{2}} - \frac{1}{90} h^{5} f^{\rb*{4}} \rb*{\zeta_{1}} &\\
    \int_{x_{0}}^{x_{3}} {f \rb*{x}} \,d x & = 
    \frac{3h}{8} \rb*{f_{0} + 3 f_{1} + 3 f_{2} + f_{3}} - \frac{3}{80} h^{5} f^{\rb*{4}} \rb*{\zeta_{1}} &\\
    \int_{x_{0}}^{x_{4}} {f \rb*{x}} \,d x & = 
    \frac{2h}{45} \rb*{7 f_{0} + 32 f_{1} + 12 f_{2} + 32 f_{3} + 7 f_{4}} - \frac{8 h^{7}}{945} f^{\rb*{6}} \rb*{ \zeta_{1}} &\\
    \int_{x_{0}}^{x_{5}} {f \rb*{x}} \,d x & = 
    \frac{5 h}{288} \rb*{19 f_{0} + 75 f_{1} + 50 f_{2} + 50 f_{3} + 75 f_{4} + 19 f^{5}} - \frac{275 h^{7}}{12096} f^{\rb*{6}} \rb*{\zeta_{1}}
\end{align}

\begin{sol} 1 \end{sol}

Using \textit{Numerical Mathematics and Computing by Cheney et al.} 6th ed. and consulting Wikipedia for an equivalent relation shows that the Simpson's \(\frac{1}{3}\) rule for \(n\) subintervals is given by the form
\begin{align*}
    \int_{a = x_{0}}^{b = x_{n}} { f \rb*{x}} \,d x & \approx
    \frac{h}{3} \sum_{i = 1}^{n/2} { f_{2i -2} + 4 f_{2i-1} + f_{2i}} &\\
\end{align*}

Similar to Aitken's acceleration and Richardson extrapolation, one can then use two approximations that were calculated with different intervals to calculate a more accurate approximation. 

Using only two intervals yields the familiar formulation for the first order Simpson's \(\frac{1}{3}\) rule:
\begin{align*}
    \int_{a}^{b} {f \rb*{x}} \,d x & \approx
    \frac{h}{3} \sum_{i = 1}^{2 / 2} {f_{2i-2} + 4 f_{2i-1} + f_{2i}} &\\
& = \frac{h}{3} \rb*{f_{0} + 4 f_{1} + f_{2} }&\\
\end{align*}

or if we adjust the \(h \rightarrow \frac{h}{2}\) and  adjust the \(x\) values to reflect the altered interval: 
\begin{align*}
    \int_{x_{0}}^{x_{4}} {f \rb*{x}} \,d x & = 
    \frac{2 h}{3} \rb*{f_{0} + 4 f_{2} + f_{4}} &\\
\end{align*}

Dividing the two intervals in half creates four subintervals, resulting in an approximation of the following form:
\begin{align*}
    \int_{x_{0}}^{x_{4}} {f \rb*{x}} \,d  x & = 
    \frac{h}{3} \sum_{i = 1}^{4 / 2} {f_{2i - 2} + 4 f_{2 i -1} + f_{2i}} = 
    \frac{h}{3} \sum_{i = 1}^{ 2} {f_{2i - 2} + 4 f_{2 i -1} + f_{2i}}  &\\
    & = \frac{h}{3} \sqrb*{\rb*{
            f_{0} + 4 f_{1} + f_{2}
    } + \rb*{
        f_{2} + 4 f_{3} + f_{4}
}} &\\
& = \frac{h}{3} \rb*{f_{0} + 4 f_{1} + 2 f_{2} + 4 f_{3} + f_{4}} &\\
\end{align*}

Approximations using the Romberg algorithm can then be carried out using the following formula

\begin{align*}
    R \rb*{n,m} & =  R \rb*{n, m -1 } 
    + \frac{1}{4 ^{m} - 1} \sqrb*{R \rb*{n,m -1} - R \rb*{n - 1, m - 1}}. &\\
\end{align*}

where \(R \rb*{n,0}\) denotes the result with \(2^{n}\) subintervals. 

(Note: See \textit{Numerical Mathematics and Computing by Cheney et al.} 6th ed. Pg. 205 for details)

We procede by setting \(R\rb*{1,1} \) equal to Simpson's \(\frac{1}{3} \) rule, and \(R \rb*{2,1}\) equal to the Simpson's \(\frac{1}{3}\) rule with halved intervals (note than \(m = 1\) since \(m = 0 \) corresponds to the lower order integral approximation, the trapezoidal rule):
\begin{align*}
    R \rb*{2,2} & =  R \rb*{2,1} + \frac{1}{4^{2} -1 } \sqrb*{ R \rb*{2, 1} - R \rb*{1,1}} &\\
    & = \frac{h}{3} \rb*{ f_{0} + 4 f_{1} + 2 f_{2} + 4 f_{3} + f_{4}}  &\\
    & \hspace{20 pt}+ \frac{1}{15} \sqrb*{ \frac{h}{3} \rb*{ f_{0} + 4f_{1} + 2 f_{2} + 4 f_{3} + f_{4}}
- \frac{2 h}{3} \rb*{f_{0} + 4 f_{2} + f_{4}}} &\\
& = \frac{2 h}{45} \rb*{ 7 f_{0} + 32 f_{1} + 12 f_{2} + 32 f_{3} + 7 f_{4}} &\\
\end{align*}

The obtained formula correspond to \textbf{formula 4} in the given table. 

\newpage

\begin{ex}
    2
\end{ex}

In \texttt{problem 2} we are calculate the integral
\begin{align*}
    f \rb*{x} & =  \frac{2^{x} \sin \rb*{x} }{x}
\end{align*}
using the following three numerical techniques
\begin{enumerate}[label=(\alph*)]
    \item Simpson's \(\frac{1}{3}\) rule
    \item Simpson's \(\frac{1}{3}\) rule with the Romberg algorithm
    \item Gauss-Legendre with four points (data can be taken from either table 2 in the lecture notes or from page 28)
\end{enumerate}

\begin{sol}  \end{sol}

To solve this exercise a C program was created (see the folder \texttt{problem2}) with the name \texttt{integratorinator}. Executing this program with the appropriate command line arguments calculates the integral approximation using the appropriate scheme. Integrals were calculated on the interval \(x_{i} = 0.1\) to \(x_{f} = 1.1\).

Results were compared to an approximation given by Mathematica, yielding 
\begin{align*}
    I & \approx  \!1.41891918440222 &\\
\end{align*}

\begin{enumerate}[label=(\alph*)]
    \item Simpson's \(\frac{1}{3}\) rule used the following integration approximation:

        \begin{align*}
            \int_{x_{0}}^{x_{2}} {f \rb*{x}} \,d x & \approx
            \frac{h}{3} \rb*{f_{0} + 4 f_{1} + f_{2}}  &\\
        \end{align*}
Running the program yielded
\begin{align*}
    I & \approx 1.418709 &\\
\end{align*}
which differs from the answer given by Mathematica by about \(-0.0148424239 \%\).

    \item Simpson's \(\frac{1}{3}\) rule combined with the Rhomberg algorithm used the following integral approximation (see \texttt{problem 1} ).
        \begin{align*}
            \int_{x_{0}}^{x_{4}} {f \rb*{x}} \,d x & \approx
            \frac{2 h }{45} \rb*{ 7 f_{0} + 32 f_{1} + 12 f_{2} + 32 f_{3} + 7 f_{4}} &\\
        \end{align*}
        Running the program yielded
        \begin{align*}
            I & \approx 1.41892 &\\
        \end{align*}
        which differs from Mathematica's calculated value by about \(0.0000248637 \%\).
    \item Sample

\end{enumerate} 

\end{document}

\documentclass[12pt]{article}

\usepackage{breqn}
\usepackage[margin=1in]{geometry} 
\usepackage{amsmath,amsthm,amssymb,enumitem}
\usepackage[german,spanish,english]{babel}
\usepackage{tensor}
\usepackage{graphicx}
\usepackage{esint}
\usepackage[T1]{fontenc}
\usepackage{mathtools}
\usepackage{siunitx}
\newenvironment{ex}[2][Exercise]{\begin{trivlist}
\item[\hskip \labelsep {\bfseries #1}\hskip \labelsep {\bfseries #2.}]}{\end{trivlist}}

\newenvironment{sol}[1][Solution]{\begin{trivlist}
\item[\hskip \labelsep {\bfseries #1:}]}{\end{trivlist}}

\newcommand{\meq}{\overset{!}{=}}
\DeclarePairedDelimiter\bra{\langle}{\rvert}
\DeclarePairedDelimiter\ket{\lvert}{\rangle}
\DeclarePairedDelimiterX\brk[2]{\langle}{\rangle}{#1\,\delimsize\vert\,\mathopen{}#2}
\DeclareSIUnit\angstrom{\text {Å}}

%DECLARATION OF DELIMITERS%

\DeclarePairedDelimiter\vb{\lvert}{\rvert}
\DeclarePairedDelimiter\rb{(}{)}
\DeclarePairedDelimiter\sqrb{[}{]}
\DeclarePairedDelimiter\cb{\{}{\}}
\DeclarePairedDelimiter\ab{\langle}{\rangle}
\DeclarePairedDelimiter\db{\|}{\|}


\begin{document}
\noindent Richard Abele \hfill \today \\[30pt]
\centerline{ \Large{ \textbf{ Numerical Methods - Problem Set 6 }}}

\begin{ex}
    1 
\end{ex}
In \texttt{problem 1} we are asked to determine which formula from \texttt{table 1} in the lecture notes on numerical integration will analytically be derived from combining the Simpson \(\frac{1}{3}\) integration method with the Romberg integration method. 

The five formulas in the lecture note table are as follows: 

\begin{align}
    \int_{x_{0}}^{x_{1}} {f \rb*{x}} \,d x & = \frac{h}{2} \rb*{f_{0} + f_{1}} - \frac{1}{12} h^{3} f^{\rb*{2}} \rb*{ \zeta_{1}} &\\
    \int_{x_{0}}^{x_{2}} {f \rb*{x}} \,d x & = 
    \frac{h}{3} \rb*{f_{0} + 4 f_{1} + f_{2}} - \frac{1}{90} h^{5} f^{\rb*{4}} \rb*{\zeta_{1}} &\\
    \int_{x_{0}}^{x_{3}} {f \rb*{x}} \,d x & = 
    \frac{3h}{8} \rb*{f_{0} + 3 f_{1} + 3 f_{2} + f_{3}} - \frac{3}{80} h^{5} f^{\rb*{4}} \rb*{\zeta_{1}} &\\
    \int_{x_{0}}^{x_{4}} {f \rb*{x}} \,d x & = 
    \frac{2h}{45} \rb*{7 f_{0} + 32 f_{1} + 12 f_{2} + 32 f_{3} + 7 f_{4}} - \frac{8 h^{7}}{945} f^{\rb*{6}} \rb*{ \zeta_{1}} &\\
    \int_{x_{0}}^{x_{5}} {f \rb*{x}} \,d x & = 
    \frac{5 h}{288} \rb*{19 f_{0} + 75 f_{1} + 50 f_{2} + 50 f_{3} + 75 f_{4} + 19 f^{5}} - \frac{275 h^{7}}{12096} f^{\rb*{6}} \rb*{\zeta_{1}}
\end{align}

\begin{sol} 1 \end{sol}

Using \textit{Numerical Mathematics and Computing by Cheney et al.} 6th ed. and consulting Wikipedia for an equivalent relation shows that the Simpson's \(\frac{1}{3}\) rule for \(n\) subintervals is given by the form
\begin{align*}
    \int_{a = x_{0}}^{b = x_{n}} { f \rb*{x}} \,d x & \approx
    \frac{h}{3} \sum_{i = 1}^{n/2} { f_{2i -2} + 4 f_{2i-1} + f_{2i}} &\\
\end{align*}

Similar to Aitken's acceleration and Richardson extrapolation, one can then use two approximations that were calculated with different intervals to calculate a more accurate approximation. 

Using only two intervals yields the familiar formulation for the first order Simpson's \(\frac{1}{3}\) rule:
\begin{align*}
    \int_{a}^{b} {f \rb*{x}} \,d x & \approx
    \frac{h}{3} \sum_{i = 1}^{2 / 2} {f_{2i-2} + 4 f_{2i-1} + f_{2i}} &\\
& = \frac{h}{3} \rb*{f_{0} + 4 f_{1} + f_{2} }&\\
\end{align*}

or if we adjust the \(h \rightarrow \frac{h}{2}\) and  adjust the \(x\) values to reflect the altered interval: 
\begin{align*}
    \int_{x_{0}}^{x_{4}} {f \rb*{x}} \,d x & = 
    \frac{2 h}{3} \rb*{f_{0} + 4 f_{2} + f_{4}} &\\
\end{align*}

Dividing the two intervals in half creates four subintervals, resulting in an approximation of the following form:
\begin{align*}
    \int_{x_{0}}^{x_{4}} {f \rb*{x}} \,d  x & = 
    \frac{h}{3} \sum_{i = 1}^{4 / 2} {f_{2i - 2} + 4 f_{2 i -1} + f_{2i}} = 
    \frac{h}{3} \sum_{i = 1}^{ 2} {f_{2i - 2} + 4 f_{2 i -1} + f_{2i}}  &\\
    & = \frac{h}{3} \sqrb*{\rb*{
            f_{0} + 4 f_{1} + f_{2}
    } + \rb*{
        f_{2} + 4 f_{3} + f_{4}
}} &\\
& = \frac{h}{3} \rb*{f_{0} + 4 f_{1} + 2 f_{2} + 4 f_{3} + f_{4}} &\\
\end{align*}

Approximations using the Romberg algorithm can then be carried out using the following formula

\begin{align*}
    R \rb*{n,m} & =  R \rb*{n, m -1 } 
    + \frac{1}{4 ^{m} - 1} \sqrb*{R \rb*{n,m -1} - R \rb*{n - 1, m - 1}}. &\\
\end{align*}

where \(R \rb*{n,0}\) denotes the result with \(2^{n}\) subintervals. 

(Note: See \textit{Numerical Mathematics and Computing by Cheney et al.} 6th ed. Pg. 205 for details)

We procede by setting \(R\rb*{1,1} \) equal to Simpson's \(\frac{1}{3} \) rule, and \(R \rb*{2,1}\) equal to the Simpson's \(\frac{1}{3}\) rule with halved intervals (note than \(m = 1\) since \(m = 0 \) corresponds to the lower order integral approximation, the trapezoidal rule):
\begin{align*}
    R \rb*{2,2} & =  R \rb*{2,1} + \frac{1}{4^{2} -1 } \sqrb*{ R \rb*{2, 1} - R \rb*{1,1}} &\\
    & = \frac{h}{3} \rb*{ f_{0} + 4 f_{1} + 2 f_{2} + 4 f_{3} + f_{4}}  &\\
    & \hspace{20 pt}+ \frac{1}{15} \sqrb*{ \frac{h}{3} \rb*{ f_{0} + 4f_{1} + 2 f_{2} + 4 f_{3} + f_{4}}
- \frac{2 h}{3} \rb*{f_{0} + 4 f_{2} + f_{4}}} &\\
& = \frac{2 h}{45} \rb*{ 7 f_{0} + 32 f_{1} + 12 f_{2} + 32 f_{3} + 7 f_{4}} &\\
\end{align*}

The obtained formula correspond to \textbf{formula 4} in the given table. 

\newpage

\begin{ex}
    2
\end{ex}

In \texttt{problem 2} we are calculate the integral
\begin{align*}
    f \rb*{x} & =  \frac{2^{x} \sin \rb*{x} }{x}
\end{align*}
using the following three numerical techniques
\begin{enumerate}[label=(\alph*)]
    \item Simpson's \(\frac{1}{3}\) rule
    \item Simpson's \(\frac{1}{3}\) rule with the Romberg algorithm
    \item Gauss-Legendre with four points (data can be taken from either table 2 in the lecture notes or from page 28)
\end{enumerate}

\begin{sol}  \end{sol}

To solve this exercise a C program was created (see the folder \texttt{problem2}) with the name \texttt{integratorinator}. Executing this program with the appropriate command line arguments calculates the integral approximation using the appropriate scheme. Integrals were calculated on the interval \(x_{i} = 0.1\) to \(x_{f} = 1.1\).

Results were compared to an approximation given by Mathematica, yielding 
\begin{align*}
    I & \approx  \!1.41891918440222 &\\
\end{align*}

\begin{enumerate}[label=(\alph*)]
    \item Simpson's \(\frac{1}{3}\) rule used the following integration approximation:

        \begin{align*}
            \int_{x_{0}}^{x_{2}} {f \rb*{x}} \,d x & \approx
            \frac{h}{3} \rb*{f_{0} + 4 f_{1} + f_{2}}  &\\
        \end{align*}
Running the program yielded
\begin{align*}
    I & \approx 1.418709 ,&\\
\end{align*}
which differs from the answer given by Mathematica by about \(-0.0148424239 \%\).

    \item Simpson's \(\frac{1}{3}\) rule combined with the Rhomberg algorithm used the following integral approximation (see \texttt{problem 1} ).
        \begin{align*}
            \int_{x_{0}}^{x_{4}} {f \rb*{x}} \,d x & \approx
            \frac{2 h }{45} \rb*{ 7 f_{0} + 32 f_{1} + 12 f_{2} + 32 f_{3} + 7 f_{4}} &\\
        \end{align*}
        Running the program yielded
        \begin{align*}
            I & \approx 1.41892 ,&\\
        \end{align*}
        which differs from Mathematica's calculated value by about \(0.0000248637 \%\).
    \item In order to solve this integral using Gauss-Legendre quadrature, we first consider the the form that the solution needs to take. 
    \begin{align*}
        \int_{c}^{d} {f \rb*{x}} \,d x & = 
        \int_{-1}^{1} { f \rb*{x}} \,d x  \approx
        \sum_{i = 1}^{n} {A_{i} f \rb*{x_{i}}}  &\\
    \end{align*}
    With \(A_{i}\) being the quadrature weights and \(x_{i}\) being the roots of the \(n\)th Legendre polynomial.

    In order for us to use this approximation for an arbitrary interval \(\sqrb*{a,b}\), we use the following substitution:
    \begin{align*}
        \lambda \rb*{t} & = \rb*{\frac{b-a}{d-c}} t + \rb*{\frac{ad - bc}{d -c}} &\\
        dx & = \lambda' \rb*{t} d t 
         = \frac{b-a}{d-c} d t, &\\
    \end{align*}
    resulting in 
    \begin{align*}
        \int_{a}^{b} {f \rb*{x}} \,d x & = \int_{c}^{d} {f \rb*{ \lambda (t)} \frac{b-a}{d-c}} \,d t &\\
        & \approx
        \rb*{\frac{b-a}{d-c}} \sum_{i=1}^{n} {
            A_{i} f \rb*{ \sqrb*{
                    \frac{b-a}{d-c}
            }t_{i} 
        + \sqrb*{\frac{ad-bc}{d-c}}}
        }. &\\
    \end{align*}
    If we then plug in \(\sqrb*{c,d} = \sqrb*{-1,1}\), we obtain the expression
    \begin{align*}
        \int_{a}^{b} {f \rb*{x}} \,d x & \approx
        \frac{b-a}{2} \sum_{i = 1}^{n} {
            A_{i} f \rb*{\sqrb*{\frac{b-a}{2}}t_{i} + \frac{a+b}{2}}
        } &\\
    \end{align*}
    For our usecase of \(\sqrb*{a,b} = \sqrb*{0.1,1.1}\) that results in 
    \begin{align*}
        \int_{a}^{b} {f \rb*{x}} \,d x & = 
        \int_{0.1}^{1.1} {f \rb*{x}} \,d x &\\
         & \approx \rb*{0.5} \sum_{i = 1 }^{4} {A_{i} f \rb*{
                 0.5 t_{i} + 0.6
         }}
    \end{align*}
    From the lecture notes we know that for four points, the \(x\) values needed are given by
    \begin{align*}
        x_{1} & = - 0.861136 &\\
        x_{2} & = - 0.339981 &\\
        x_{3} & =  0.339981 &\\
        x_{4} & =  0.861136
    \end{align*}
    and the \(A\) values are given by 
    \begin{align*}
        A_{1} & = 0.347855 &\\
        A_{2} & = 0.652145 &\\
        A_{3} & = 0.652145 &\\
        A_{4} & = 0.347855 
    \end{align*}
    Implementing this approximation into C and running the program yields
    \begin{align*}
        I & \approx 1.41891 ,&\\
    \end{align*}
    which differs from the Mathematica-calculated value by \(0.000000180 \%\).

\end{enumerate} 
As can quite easily be seen, the accuracy of these approximations increases as one moves through the list. This is not too surprising, as the approximations use progressively more information (fist increasing the numbers of points used, and then fainlly adding weights).

\newpage

\begin{ex}
    3
\end{ex}

\texttt{Problem 3} involves numerically proving the relations given by the following three two-dimensional integrals:
\begin{align}
    I_{1} & =  \int_{0}^{1} { \rb*{\int_{0}^{2} {x y^{2}} \,d x}} \,d y  = \frac{2}{3} &\\
    I_{2} & =  \int_{0}^{1} {\rb*{ \int_{2y}^{2} {x y^{2}} \,d x}} \,d y  =  \frac{4}{15} &\\
    I_{3} & = \int_{0}^{2} { \rb*{ \int_{0}^{x/2} {x y^{2}} \,d y}} \,d x  =  ? 
\end{align}

\begin{sol}  \end{sol}

According to the lecture notes, if the limits of integration are constants, we can apply one of our previous integration techniques in sequence.

For this exercise let \(f \rb*{x, y} = x y^{2} = f_{x,y}\).

We begin by considering the general integral
\begin{align*}
    I & = \int_{a}^{b} { \rb*{\int_{c}^{d} {x y^{2}} \,d x}} \,d y  = 
    \int_{a}^{b} { \rb*{\int_{c}^{d} {f_{x,y}} \,d x}} \,d y &\\
\end{align*}
First we calculate the approximation for the inner integral using Simpson's \(\frac{1}{3}\) rule
\begin{align*}
    I_{\text{in} } & =  \int_{c}^{d} {f_{x,y}} \,d x &\\
     & \approx \frac{h_{x}}{3} \rb*{f_{0,y} + 4 f_{1,y} + f_{2,y}}, 
\end{align*}
where \(h_{x}  =  \frac{d- c}{2}\).

Placing this approximation into the outer integral yields
\begin{align*}
    I & \approx  \int_{a}^{b} {\rb*{\frac{h_{x}}{3} \rb*{f_{0,y} + 4 f_{1,y} + f_{2,y}}}} \,d y &\\
          & = \frac{h_{x}}{3} \int_{a}^{b} {\rb*{f_{0,y} + 4 f_{1,y} + f_{2,y}}} \,d y. &\\
\end{align*}
If we then once again apply Simpson's \(\frac{1}{3}\) rule, we obtain the following numericall approximation for the integral:
\begin{align*}
    I & \approx \frac{h_{x}}{3} \frac{h_{y}}{3} \sqrb*{
        \rb*{f_{0,0} + 4 f_{1,0} + f_{2,0}} +
        4 \rb*{f_{0,1} + f_{1,1} + f_{2,1}} + 
        \rb*{f_{0,2} + 4 f_{1,2} + f_{2,2}}
    } &\\
    & = \frac{h_{x} h_{y}}{9} 
    \sqrb*{
        \rb*{f_{0,0} + 4 f_{1,0} + f_{2,0}} +
        4 \rb*{f_{0,1} + f_{1,1} + f_{2,1}} + 
        \rb*{f_{0,2} + 4 f_{1,2} + f_{2,2}}
    }, &\\
\end{align*}
with \(h_{y} = \frac{b - a}{2}\).

Written out fully, we get
\begin{align*}
    I & \approx \frac{(d-c)(b-a)}{36}
    \sqrb*{
        \rb*{f_{0,0} + 4 f_{1,0} + f_{2,0}} +
        4 \rb*{f_{0,1} + f_{1,1} + f_{2,1}} + 
        \rb*{f_{0,2} + 4 f_{1,2} + f_{2,2}}
    }, &\\
\end{align*}

\begin{enumerate}[label=(\alph*)]
    \item 
Using this expression to calculate the numerical value of \(I_{1}\) yields exactly the expected result.

\item 

In order to calculate the value of \(I_{2}\), we go from this expression obtained during the calculation of \(I\)
\begin{align*}
    I_{1} & \approx  \int_{a}^{b} {\rb*{\frac{h_{x}}{3} \rb*{f_{0,y} + 4 f_{1,y} + f_{2,y}}}} \,d y . &\\
\end{align*}
Due to the altering of \(c\), we must change the value of \(h_{x}\) as follows:
\begin{align*}
    h_{x} & = \frac{d - c}{2} = \frac{d - 2 y}{2} = h_{x} \rb*{y}&\\
\end{align*}

This results in the general approximation of 
\begin{align*}
    I_{2} & \approx  \int_{a}^{b} {\rb*{\frac{h_{x} \rb*{y}}{3} \rb*{f_{0,y} + 4 f_{1,y} + f_{2,y}}}} \,d y  &\\
    I_{2} & \approx \frac{h_{y}}{9} [
        h_{x}\rb*{y_{0}} \rb*{f_{0,0} + 4 f_{1,0} + f_{2,0}} +
        4 h_{x}\rb*{y_{1}} \rb*{f_{0,1} + f_{1,1} + f_{2,1}} &\\
& \hspace{200 px} + 
        h_{x}\rb*{y_{2}} \rb*{f_{0,2} + 4 f_{1,2} + f_{2,2}}
    ] &\\
\end{align*}
% This thus results in 
% \begin{align*}
%     I_{2} & \approx  \int_{a}^{b} {\rb*{\frac{d - 2 y}{6} \rb*{f_{0,y} + 4 f_{1,y} + f_{2,y}}}} \,d y . &\\
% \end{align*}

\end{enumerate}

\end{document}
